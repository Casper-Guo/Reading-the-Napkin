\chapter{Groups}

\section{Definition and examples of groups}

\begin{definition}
  A group is a pair consisting of a set $G$ and a binary operation $\star$ on $G$ such that:
  \begin{itemize}
    \item $G$ has an identity element
    \item The operation $\star$ is associative
    \item Every element $g \in G$ has an inverse
  \end{itemize}
  Closure of the set $G$ under operation $\star$ is implied.
\end{definition}

It is not required that $\star$ is commutative. The group is \textbf{abelian} if the binary operation is commutative and \textbf{non-abelian} otherwise.

\begin{definition}
  The cyclic group of order $n$ is the group under addition of the remainders modulo $n$ where $n \in \Z \; n > 1$.
\end{definition}

\begin{notation}
  The cyclic group of order $n$ is denoted $\Zc n$.
\end{notation}

\begin{notation}
  Let $p$ be a prime, denote the nonzero residues modulo $p$ as $(\Zc p)^\times$
\end{notation}

\begin{solution}
  Question 1.1.10: Consider the pair $(\Zm 4, \times)$, note that 2 has no inverse. For any non-prime $p$, there must exist another integer $x$ such that $\gcd{x, p} \neq 1$ and thus $x$ has no inverse.
\end{solution}

\begin{solution}
  Question 1.1.16: The identity is $(1_g, 1_h)$. The inverse of $(g, h)$ is $(g^{-1}, h^{-1})$ as $(g, h) \cdot (g^{-1}, h^{-1}) = (g \star g^{-1}, h \ast h^{-1}) = (1_g, 1_h)$.
\end{solution}

\begin{solution}
  Exercise 1.1.18:
  \begin{enumerate}[label = {(\alph*)}]
    \item This is a group. The identity is 0. The inverse of $x$ is $-x$. The closure condition is met. Consider two elements $\frac{a}{b}$ and $\frac{c}{d}$, their sum is $\frac{ad + bc}{bd}$ where $bd$ is odd. So any simplification of $\frac{ad + bc}{bd}$ retains an odd denominator.
    \item This is a group. The identity is 0. The inverse of $x$ is $-x$.
    \item This is not a group as it is not closed. See counterexample $\frac{1}{2} \times \frac{1}{2} = \frac{1}{4}$ which is not in the set.
    \item This is not a group. See counterexample of $1$ having no inverse.
  \end{enumerate}
\end{solution}

\section{Properties of groups}
\begin{proof}
  Exercise 1.2.6: Bijectivity is obvious as the map $x \mapsto gx$ has the inverse map $y \mapsto g^{-1}y$.

  Suppose $gx = gy$, then multiplying both sides on the left by $g^{-1}$ gives $x = y$. Thus the map is injective.

  Consider any $y \in G$, then there exists $x \in G$ such that $x = g^{-1}y$. Thus the map is surjective.
\end{proof}

\section{Isomorphisms}
\begin{definition}
  A bijection $\phi \colon G \to H$ is called an isomorphism if:

  \[ \phi(g_1 \star g_2) = \phi(g_1) \ast \phi(g_2) \quad \text{for all $g_1, g_2 \in G$} \]
\end{definition}

\begin{solution}
  Exercise 1.3.5: Let the primitive root modulo $p$ be $r$. We use the same bijection $\phi(a \mod p - 1) = r^a \mod p$. Notice that this map is a bijection by the definition of a primitive root. Then we check $\phi(a \star b) = \phi(a) \ast \phi(b)$, which is $p^{a + b \mod p -1} \equiv p^{a \mod p - 1}p^{b \mod p - 1} \mod p$. Thus the two groups are isomorphic.
\end{solution}

$\cong$ (being isomoprhic) is an equivalence relation because:

\begin{itemize}
  \item Reflexive: Any group is isomorphic to itself via the identity map
  \item Symmetric: An isomorphism is a bijection, so there must be a reverse map
  \item Transitive: Since isomorphisms are just bijective functions, they can be composed
\end{itemize}

\section{Orders of groups, and Lagrange's theorem}

\section{Subgroups}

\section{Groups of small orders}

\section{Problems}